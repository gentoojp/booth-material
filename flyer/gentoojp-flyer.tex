\documentclass[10pt,foldmark,notumble]{leaflet}
\renewcommand*\foldmarkrule{.3mm}
\renewcommand*\foldmarklength{5mm}

\usepackage{mathptmx}
\usepackage{url}
\usepackage{ascmac}
\usepackage{fancybox}
\usepackage{graphicx}
\usepackage{xcolor}

% NOTE: Copy from "2. Main Gentoo web colors"
%       http://www.gentoo.org/proj/en/desktop/artwork/colors.xml
\definecolor{GentooDarkPurple}{HTML}{54487A}
\definecolor{GentooPurple}{HTML}{61538D}
\definecolor{GentooLightPurple}{HTML}{6E56AF}
\definecolor{GentooGray}{HTML}{DDDAEC}
\definecolor{GentooLighterPurple}{HTML}{DDDDFF}
\definecolor{GentooGreen}{HTML}{73D216}
\definecolor{GentooBlack}{HTML}{000000}
\definecolor{GentooWhite}{HTML}{FAFAFA}
\definecolor{GentooLightGreen}{HTML}{DFF0D8}
\definecolor{GentooLightYellow}{HTML}{FCF8E3}
\definecolor{GentooLightRed}{HTML}{F2DEDE}

\title{\bf Get Gentoo!}
\author{\bf GentooJP}
\date{\bf 11月5日 -- 11月6日, 2016 }

% FIXME: \includesvg does not works properly on my LaTeX env...
\AddToBackground{1}{
	\put(0,195){\includegraphics[bb=520 0 1920 1200,clip=true,scale=0.2]{gentoo-10-1920x1200.jpg}}}
\AddToBackground{1}{
	\put(115,530){\includegraphics[scale=0.1]{ripples-gblend.eps}}}
\AddToBackground{1}{
	\put(0,-200){\includegraphics[bb=-0 -400 2000 412, clip=true,scale=0.5]{emerge.jpg}}}
\AddToBackground{2}{
	\put(50,200){\includegraphics[scale=0.6]{larry-the-cow-full-udder.eps}}}
\AddToBackground{2}{
	\put(70,400){\bf \large Have you mooed today?}}
\AddToBackground{3}{
	\put(50,30){\includegraphics[bb=0 0 680 640, scale=0.95]{gentwoo_logo.eps}}}
\AddToBackground{4}{
	\put(15,150){Walbrixは、Gentoo LinuxとGenTwooを応援しています。}}
\AddToBackground{4}{
	\put(20,93){\includegraphics[scale=0.42,clip=true]{walbrix_logo.eps}}}
\AddToBackground{6}{
	\put(0,0){\textcolor{GentooLighterPurple}{\rule{\paperwidth}{\paperheight}}}}
\AddToBackground*{2}{
	\put(\LenToUnit{.5\paperwidth},\LenToUnit{.5\paperheight}){%
		\makebox(0,0)[c]{%
			\resizebox{.9\paperwidth}{!}{\rotatebox{35.26}{%
				\textsf{\textbf{\textcolor{GentooGray}{Get Gentoo!}}}}}}}}

\CutLine*{1}
\CutLine*{6}

\begin{document}
\maketitle

% FIXME: Only page 1 shows page number, dunno why...
\thispagestyle{empty}

{\centering
	\textcolor{GentooDarkPurple}{
		{\bf \large Gentoo} \\
		{コンパイルしつづけて、10年。} \\
		{\small ミラーサイトから無料でダウンロードできます。} \\
		\vspace{2mm}
		\ovalbox{{\small 今すぐインストール}} \\
	}
	\vspace{70mm}
	\textcolor{GentooWhite}{
		{\bf \large Portage} \\
		{最も先進的なパッケージの} \\
		{マネジメントシステム。} \\
		{\small
			私たちはすべてのGentooユーザーに、
			最新のバージョン,最先端のテクノロジー、最も堅牢なセキュリティーを体験してもらいたいと考えています。
			そして私たちはそれを、Portageツリーをローリングリリースすることで可能にしました。
			しかもミラーサイトからsyncできるので、手に入れるのはこれ以上なく簡単です。
			Portageは、Gentooにも、Gentooを使うすべての人にも、さらなる大きな進化をもたらします。
			新機能を見る〉
		}
	}
}
\newpage
{\centering
	\textcolor{GentooBlack}{
		{\bf \large USEフラグ} \\
		{すべては、あなたの選択。} \\
		{\small
			Gentooのパッケージマネジメントシステムは、
			あなたの選択ですべてのコンピュータと
			ソフトウェアに搭載されているテクノロジーを
			最大限に活用できるように設計されています。
			だから、すべてがあなたの思い通りに機能するのです。
			Portageツリーには、様々なことができる、
			多くのパワフルなパッケージが収録されています。
			しかも、これらのパッケージの依存関係もしっかり管理します。
			さらに詳し.
			https://goo.gl/acA4Cx
		} \\
	}
}
\newpage
{\centering
	\textcolor{GentooBlack}{
		{\bf \large GenTwoo} \\
		{ソーシャルでつながる、ビルドクラスタ。} \\
		{\small
			Gentooを使っていると、ふと思う。
			「新しいバージョンがパッケージされたけど、みんなは使っているかな?」
			「コンパイルが失敗したけど、これってわたしだけ?」
			GenTwooを使えば、それはきっと楽しい話題に変わります。
			さらに詳しく〉
		} \\
	}
}
\newpage
\section{GenTwooの使い方}
	\begin{enumerate}
		\item Betagarden Overlayを追加します。 \\
			Overlayについてわからなければ、
			Gentoo Overlay: ユーザーズ・ガイド
			\footnote{\url{https://wiki.gentoo.org/wiki/Project:Overlays/Overlays_guide}}
			\footnote{\url{https://wiki.gentoo.org/wiki/Project:Overlays/Overlays_guide}}
			を参照してください。
		\item \verb|emerge app-portage/gentwoo|を実行します。 \\
			GenTwooのスクリプトと設定ファイルがインストールされます。
		\item GenTwooの設定ファイル(\verb|/etc/gentwoo.conf|)を編集します。
			\begin{itemize}
				\item \verb|USER| \\
					TwitterのログインIDを設定します。
				\item \verb|TOKEN| \\
					token
					\footnote{\url{http://gentwoo.elisp.net/my/key}}
					を設定します。
				\item \verb|UPLOAD_LOG| \\
					Portageの簡単なログをアップロードするかどうかを設定します。
					(\verb|True|または\verb|False|)
				\item \verb|UPLOAD_ERROR_LOG| \\
					Portageがエラーした時に完全なエラーログをアップロードするかどうかを設定します。
					(\verb|True|または\verb|False|)
			\end{itemize}
		\item PortageからGenTwooのスクリプトが起動されるように
			Portageの設定ファイル(/etc/make.conf)を編集します。
			\begin{itemize}
				\item \verb|PORTAGE_ELOG_CLASSES="log warn error qa"|
				\item \verb|PORTAGE_ELOG_SYSTEM="custom:* echo mail"|
				\item \verb|PORTAGE_ELOG_COMMAND="/usr/bin/gentwoo \| \\
					\hspace{25mm} \verb|'${PACKAGE}' '${LOGFILE}'"|
			\end{itemize}
		\item あとはコンパイルするだけ!!
	\end{enumerate}
\newpage
\section{リンク}
\begin{itemize}
	\item Gentoo.org \\
		\url{http://www.gentoo.org}
		\begin{itemize}
			\item Gentoo Linux Graphics \\
				\url{http://www.gentoo.org/main/en/graphics.xml}
		\end{itemize}
	\item GentooJP \\
		\url{http://www.gentoo.jp}
		\begin{itemize}
			\item スポンサー \\
				株式会社びぎねっと \\
				\url{http://Begi.net}
			\item OSC2016 Tokyo/Fall 展示, Gentoo勉強会の主催 \\
				\verb|@aliceinwire| \\
				\url{http://twitter.com/aliceinwire}
			\item サーバー管理 \\
				\verb|@nabeken|, \verb|@kjm| \\
				\url{http://twitter.com/nabeken} \\
				\url{http://twitter.com/kjm}
			\item エラい人 \\
				\verb|@masatomon|, \verb|@matsuu| \\
				\url{http://twitter.com/masatomon} \\
				\url{http://twitter.com/matsuu}
			\item チラシ \\
				\verb|@hiyuh|, \verb|@aliceinwire| \\
				\url{http://twitter.com/hiyuh}
		\end{itemize}
	\item GenTwoo \\
		\url{http://gentwoo.elisp.net}
		\begin{itemize}
			\item スポンサー \\
				Gentoo Linuxと Xenをベースに作られた日本語だけで操作できる仮想化OS Walbrix \\
				\url{http://walbrix.net}
			\item 作った人 \\
				\verb|@naota344| \\
				\url{http://twitter.com/naota344}
			\item レポジトリ \\
				\url{https://github.com/naota/gentwoo}
			\item アイコン \\
				\verb|@unarist| \\
				\url{http://twitter.com/unarist}
		\end{itemize}
\end{itemize}

\end{document}